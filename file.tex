\documentclass[12pt, letterpaper]{article}
\usepackage[utf8]{inputenc}
\usepackage{graphicx}
\usepackage{parskip}
\graphicspath{/}

\centering
\title{Introduction to LaTeX}
\author{Ryan Cunningham \thanks{To my mentors, friends and family}}
\date{\today}

\begin{document}

\maketitle

\begin{abstract}

Though the length of this might not warrant an abstract, this document will serve as a brief introduction to LaTeX.
	
\end{abstract}

\section{What's LaTeX?}


Let \textbf{this} sentence illustrate to you that, \textit{first} of \textbf{all}, \underline{LateX} is \emph{pretty} cool!

\subsection{Backstory on LaTeX}

\paragraph{What's \LaTeX \space again?}
LaTeX is a software system for document preparation that gives users a convenient way to format and display plain text in a highly configurable, modular, and adjustable manner.

\begin{figure}[h]
\centering
\includegraphics[width=.12\textwidth]{latex}
\caption{The Latex Project has a bird logo}
\label{birdlogo}
\end{figure}
Above is the gorgeous bird logo of the LaTeX project (figure \ref{birdlogo}). \newline


% comment
LaTeX can be helpful for: 
\begin{enumerate}
	\item{Organizing thoughts}	
	\item{Resumes}
	\item{Mathematical documents, expressing equations like $E=mc^2$} \newline

\end{enumerate} 
Second equation example:
	\[ \pi r^2 = circumference\]


\end{document}